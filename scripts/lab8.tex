% Options for packages loaded elsewhere
\PassOptionsToPackage{unicode}{hyperref}
\PassOptionsToPackage{hyphens}{url}
%
\documentclass[
  english,
  man]{apa6}
\usepackage{lmodern}
\usepackage{amssymb,amsmath}
\usepackage{ifxetex,ifluatex}
\ifnum 0\ifxetex 1\fi\ifluatex 1\fi=0 % if pdftex
  \usepackage[T1]{fontenc}
  \usepackage[utf8]{inputenc}
  \usepackage{textcomp} % provide euro and other symbols
\else % if luatex or xetex
  \usepackage{unicode-math}
  \defaultfontfeatures{Scale=MatchLowercase}
  \defaultfontfeatures[\rmfamily]{Ligatures=TeX,Scale=1}
\fi
% Use upquote if available, for straight quotes in verbatim environments
\IfFileExists{upquote.sty}{\usepackage{upquote}}{}
\IfFileExists{microtype.sty}{% use microtype if available
  \usepackage[]{microtype}
  \UseMicrotypeSet[protrusion]{basicmath} % disable protrusion for tt fonts
}{}
\makeatletter
\@ifundefined{KOMAClassName}{% if non-KOMA class
  \IfFileExists{parskip.sty}{%
    \usepackage{parskip}
  }{% else
    \setlength{\parindent}{0pt}
    \setlength{\parskip}{6pt plus 2pt minus 1pt}}
}{% if KOMA class
  \KOMAoptions{parskip=half}}
\makeatother
\usepackage{xcolor}
\IfFileExists{xurl.sty}{\usepackage{xurl}}{} % add URL line breaks if available
\IfFileExists{bookmark.sty}{\usepackage{bookmark}}{\usepackage{hyperref}}
\hypersetup{
  pdftitle={Practicing GitHub: Lab 8},
  pdfauthor={Sarah Spafford1, Heather Terral1, \& Maggie Head1},
  pdfkeywords={GitHub, practice, abstract},
  hidelinks,
  pdfcreator={LaTeX via pandoc}}
\urlstyle{same} % disable monospaced font for URLs
\usepackage{graphicx,grffile}
\makeatletter
\def\maxwidth{\ifdim\Gin@nat@width>\linewidth\linewidth\else\Gin@nat@width\fi}
\def\maxheight{\ifdim\Gin@nat@height>\textheight\textheight\else\Gin@nat@height\fi}
\makeatother
% Scale images if necessary, so that they will not overflow the page
% margins by default, and it is still possible to overwrite the defaults
% using explicit options in \includegraphics[width, height, ...]{}
\setkeys{Gin}{width=\maxwidth,height=\maxheight,keepaspectratio}
% Set default figure placement to htbp
\makeatletter
\def\fps@figure{htbp}
\makeatother
\setlength{\emergencystretch}{3em} % prevent overfull lines
\providecommand{\tightlist}{%
  \setlength{\itemsep}{0pt}\setlength{\parskip}{0pt}}
\setcounter{secnumdepth}{-\maxdimen} % remove section numbering
% Make \paragraph and \subparagraph free-standing
\ifx\paragraph\undefined\else
  \let\oldparagraph\paragraph
  \renewcommand{\paragraph}[1]{\oldparagraph{#1}\mbox{}}
\fi
\ifx\subparagraph\undefined\else
  \let\oldsubparagraph\subparagraph
  \renewcommand{\subparagraph}[1]{\oldsubparagraph{#1}\mbox{}}
\fi
% Manuscript styling
\usepackage{upgreek}
\captionsetup{font=singlespacing,justification=justified}

% Table formatting
\usepackage{longtable}
\usepackage{lscape}
% \usepackage[counterclockwise]{rotating}   % Landscape page setup for large tables
\usepackage{multirow}		% Table styling
\usepackage{tabularx}		% Control Column width
\usepackage[flushleft]{threeparttable}	% Allows for three part tables with a specified notes section
\usepackage{threeparttablex}            % Lets threeparttable work with longtable

% Create new environments so endfloat can handle them
% \newenvironment{ltable}
%   {\begin{landscape}\begin{center}\begin{threeparttable}}
%   {\end{threeparttable}\end{center}\end{landscape}}
\newenvironment{lltable}{\begin{landscape}\begin{center}\begin{ThreePartTable}}{\end{ThreePartTable}\end{center}\end{landscape}}

% Enables adjusting longtable caption width to table width
% Solution found at http://golatex.de/longtable-mit-caption-so-breit-wie-die-tabelle-t15767.html
\makeatletter
\newcommand\LastLTentrywidth{1em}
\newlength\longtablewidth
\setlength{\longtablewidth}{1in}
\newcommand{\getlongtablewidth}{\begingroup \ifcsname LT@\roman{LT@tables}\endcsname \global\longtablewidth=0pt \renewcommand{\LT@entry}[2]{\global\advance\longtablewidth by ##2\relax\gdef\LastLTentrywidth{##2}}\@nameuse{LT@\roman{LT@tables}} \fi \endgroup}

% \setlength{\parindent}{0.5in}
% \setlength{\parskip}{0pt plus 0pt minus 0pt}

% \usepackage{etoolbox}
\makeatletter
\patchcmd{\HyOrg@maketitle}
  {\section{\normalfont\normalsize\abstractname}}
  {\section*{\normalfont\normalsize\abstractname}}
  {}{\typeout{Failed to patch abstract.}}
\patchcmd{\HyOrg@maketitle}
  {\section{\protect\normalfont{\@title}}}
  {\section*{\protect\normalfont{\@title}}}
  {}{\typeout{Failed to patch title.}}
\makeatother
\shorttitle{Practicing GitHub}
\keywords{GitHub, practice, abstract}
\DeclareDelayedFloatFlavor{ThreePartTable}{table}
\DeclareDelayedFloatFlavor{lltable}{table}
\DeclareDelayedFloatFlavor*{longtable}{table}
\makeatletter
\renewcommand{\efloat@iwrite}[1]{\immediate\expandafter\protected@write\csname efloat@post#1\endcsname{}}
\makeatother
\usepackage{csquotes}
\ifxetex
  % Load polyglossia as late as possible: uses bidi with RTL langages (e.g. Hebrew, Arabic)
  \usepackage{polyglossia}
  \setmainlanguage[]{english}
\else
  \usepackage[shorthands=off,main=english]{babel}
\fi
\usepackage[]{biblatex}
\addbibresource{r-references.bib}

\title{Practicing GitHub: Lab 8}
\author{Sarah Spafford\textsuperscript{1}, Heather Terral\textsuperscript{1}, \& Maggie Head\textsuperscript{1}}
\date{}


\authornote{

Department of Counseling Psychology and Human Services, University of Oregon.

This is where I write an author note, but I am not sure what to say so I shall leave you with this: Peace be with you.

The authors made the following contributions. Sarah Spafford: Conceptualization, Writing - Original Draft Preparation, Writing - Review \& Editing; Heather Terral: Writing - Review \& Editing; Maggie Head: Writing - Review \& Editing.

Correspondence concerning this article should be addressed to Sarah Spafford, 123 PhD. E-mail: \href{mailto:dont@me.com}{\nolinkurl{dont@me.com}}

}

\affiliation{\vspace{0.5cm}\textsuperscript{1} University of Oregon}

\abstract{
This is typically where you would summarize your research with an intro, methods, results, and discussion. But, that would take some time. So here is a few lines to fill in the spot that would be an abstract.
}



\begin{document}
\maketitle

\textcite{benham2010} is a peer-reviewed article, as is the work by Gallo and colleagues \autocite*{gallo2014}

\hypertarget{methods}{%
\section{Methods}\label{methods}}

We report how we determined our sample size, all data exclusions (if any), all manipulations, and all measures in the study.

\hypertarget{participants}{%
\subsection{Participants}\label{participants}}

\hypertarget{material}{%
\subsection{Material}\label{material}}

\hypertarget{procedure}{%
\subsection{Procedure}\label{procedure}}

\hypertarget{data-analysis}{%
\subsection{Data analysis}\label{data-analysis}}

\begin{verbatim}
## # A tibble: 6 x 12
##   schidkn sex   frl   reg_size reg_size_aid small_size white black other totexp
##     <int> <chr> <chr>    <int>        <int>      <int> <int> <int> <int>  <int>
## 1      63 girl  no           0            0          1     1     0     0      7
## 2      20 girl  no           0            0          1     0     1     0     21
## 3      19 boy   yes          0            1          0     0     1     0      0
## 4      69 boy   no           1            0          0     1     0     0     16
## 5      79 boy   yes          0            0          1     1     0     0      5
## 6       5 boy   yes          1            0          0     1     0     0      8
## # ... with 2 more variables: tmathss <int>, treadss <int>
\end{verbatim}

\begin{verbatim}
## # A tibble: 4 x 7
## # Groups:   sex [2]
##   sex   frl       n math_mean math_sd rdg_mean rdg_sd
##   <chr> <chr> <int>     <dbl>   <dbl>    <dbl>  <dbl>
## 1 boy   no     1544      493.    46.3     441.   32.3
## 2 boy   yes    1410      470.    46.1     425.   26.6
## 3 girl  no     1429      501.    46.0     449.   34.5
## 4 girl  yes    1365      478.    46.3     431.   27.4
\end{verbatim}

We used R \autocite[Version 3.6.2;][]{R-base} and the R-packages \emph{\}dplyr} {[}@\}R-dplyr{]}, and \emph{papaja} \autocite[Version 0.1.0.9997;][]{R-papaja} for all our analyses.

\hypertarget{results}{%
\section{Results}\label{results}}

\begin{table}[tbp]

\begin{center}
\begin{threeparttable}

\caption{\label{tab:apa_table}Descriptive statistics from star data}

\begin{tabular}{lllllll}
\toprule
sex & \multicolumn{1}{c}{frl} & \multicolumn{1}{c}{n} & \multicolumn{1}{c}{math\_mean} & \multicolumn{1}{c}{math\_sd} & \multicolumn{1}{c}{rdg\_mean} & \multicolumn{1}{c}{rdg\_sd}\\
\midrule
boy & no & 1544 & 492.85 & 46.34 & 441.46 & 32.32\\
boy & yes & 1410 & 469.87 & 46.09 & 425.38 & 26.63\\
girl & no & 1429 & 501.21 & 45.96 & 448.54 & 34.52\\
girl & yes & 1365 & 477.51 & 46.30 & 430.80 & 27.42\\
\bottomrule
\addlinespace
\end{tabular}

\begin{tablenotes}[para]
\normalsize{\textit{Note.} This table was created with apa\_table().}
\end{tablenotes}

\end{threeparttable}
\end{center}

\end{table}

Girls eligible for free/reduced price lunch (FRPL) had higher math scores than boys eligible for FRPL. Girls eligible for FRPL had higher scores than boys eligble for FRPL. Girls not eligible for FRPL had, on average, higher math and reading scores than boys who were not eligible.

\hypertarget{discussion}{%
\section{Discussion}\label{discussion}}

\begingroup
\setlength{\parindent}{-0.5in}
\setlength{\leftskip}{0.5in}

\newpage

\#I'm not sure what this is doing
::: \{\#refs custom-style=\enquote{Bibliography}\}
:::

\endgroup


\printbibliography

\end{document}
